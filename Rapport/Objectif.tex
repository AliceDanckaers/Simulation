L’objectif de la présente étude est de simuler un aéroport à l’aide d’une simulation à événement discret. L'attention sera portée sur les arrivées des avions et les temps q'ils prennent à effectuer les différentes étapes, de leur arrivée dans l'espace aérien de l'aéroport jusqu'à leur départ.
La simulation devra déterminer les surcharges éventuelles de l’aéroport et leur contexte, et ce en fonction des infrastructures disponibles. Elle devra être modulable, afin de faire varier ces infrastructures (notamment en modifiant les nombres de portes d’accès et de pistes) et d’étudier les conséquences de ces variations.
Le livrable final est composé du programme de simulation (sous la forme d'un projet JAVA compilable) et d'une étude des données générées par ce programme. Cet étude permettra notamment de déterminer les moyennes de fréquentation journalière, de retards, de temps d'attentes avant atterissage, etc.
