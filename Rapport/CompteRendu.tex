%Revue critique du fonctionnement et des résultats. 
Le suivi de la qualité du code a été effectuée tout au long du développement. En premier à l'aide des fonctionnalités d’Éclipse, qui permettent à chaque instant de vérifier les dépendances entre classes, packages et le typage des entrées-sorties. Ensuite à l'aide de la fonction \textit{print\_status()}, qui affiche les états des infrastructures dans la console et peut être utilisée par exemple après chaque événement. De plus, la première version des événements affichaient les log dans la console, plutôt que de les enregistrer, ce qui a permis des tests de fonctionnement.
Enfin, même après la fin de la phase de développement, nous avons gardé un esprit critique sur nos résultats, ce qui a permis notamment de revenir sur une incohérence dans la libération de la piste lors de la phase d'atterrissage : l'avion ne libérait la piste qu'à la fin de la phase de roulage, ce qui empêchait un éventuel avion en attente de décoller durant ce laps de temps.
