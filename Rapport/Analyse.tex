%Systèmes, entités, variables, événements, processus… 

Le système que nous allons simuler est un aéroport. Toutefois, notre modèle peut-être grandement simplifié par rapport à  un aéroport réaliste. Puisque l'étude se focalise sur la surcharge du terminal, seul le processus d'atterrissage, de décollage et de traitement des avions nous intéresse. On peut donc ignorer de nombreux aspects, tels que la gestion des personnels, l'entretien des infrastructures, la gestion financière.
La seule chose pertinente dans notre cadre est le statut des avions et des infrastructures (occupés ou non) au cours du temps. c'est à partir de ces informations que nous pourront extraire les statistiques de fréquentation demandées.
Ainsi les systèmes environnant en interaction avec le système étudié sont essentiellement constitué des Avions.


Les entités de simulations identifiées à la lecture du sujet sont :
\begin{itemize}[label=--]
\item L'aéroport de Brest
\item Les avions
\item les infrastructures de l'aéroport, a savoir :
\end{itemize}

\begin{itemize}[leftmargin=3cm]
	\item la piste
	\item un taxiway utilisé à l'atterrissage
	\item plusieurs portes (4, 6 ou 8)
	\item un taxiway utilisé au décollage
\end{itemize}


Le cycle de vie des infrastructures est bien sur inclus dans celui de l'aéroport. C'est l'aéroport qui donne créé ses infrastructures. Nous allons dans les paragraphes qui suivent analyser chaque entité et identifier puis caractériser:
\begin{itemize}[label=--]
\item leurs variables d'états
\item leurs propriétaires d'initialisation et caractéristiques
\item les événements temporels, et d'états
\item les comportements et hypothèses de modélisation
\end{itemize} 

\section{L'aéroport}

\begin{table}[H]
\begin{center}
\begin{tabular}{|>{\begin{bf} \columncolor{lightblue}} p{7cm} <{\end{bf}}|p{7cm}|}
  \hline
  Entité & Aéroport \\ 
  \hline
  Type d'Entité & Permanente \\ 
  \hline
    & meteo  \\
    & currDate \\
    & waitingListLanding \\
    & waitingListGate \\
    & waitingListTW2 \\
    \multirow{-6}{7cm}{Variables d'états}& waitingListTakeOff \\
  \hline
   Variables statistiques de scrutation & \\ 
  \hline
   & startDate \\
   & stopDate \\
   & evtinit \\
   \multirow{-4}{7cm}{Paramètres techniques et données d'initialisation } & nb\_gate \\
  \hline
  	Événements & Aucun\\
  \hline
   Processus Stochastique & Aucun\\ 
   \hline
\end{tabular}
\end{center}
\caption{Analyse de l'entité aéroport}
\label{aeroAna}
\end{table}

Les données d'initialisation de l'aéroport sont les suivantes:
\begin{itemize}[label=--]
\item La date d'ouverture de l'aéroport (début de la simulation)
\item La date de fermeture de l'aéroport (fin de la simulation)
\item La météo au début de la simulation
\item Le nombre de portes
\end{itemize}
Elles sont lues dans le fichier de configuration (config.properties) et permettent de configurer l'état initial de l'aéroport.

Les états de l'aéroport sont :
\begin{itemize}[label=--]
\item Les conditions météorologique au niveau de l'aéroport, c'est-à-dire la météo
\item La date de l'événement courant
\item Une liste d'attente qui contiendra les avions qui ont contactés la tour de contrôle mais qui doivent attendre la libération de la piste et du taxiway 1 pour atterrir.
\item Une liste d'attente qui pourra contenir un avion qui attendra qu'une porte se libère sur le taxiway 1
\item Une liste d'attente qui contiendra les avions qui attendent à une porte la libération du taxiway 2 pour repartir
\item Une liste d'attente qui pourra contenir un avion qui attendra que la piste se libère sur le taxiway 2 pour décoller.
\end{itemize}

Il n'y a pas d'événements temporels qui sont directement liés avec l'aéroport, car on a mis en place l'entité infrastructure
\section{Les avions}

\begin{table}[H]
\begin{center}
\begin{tabular}{|>{\begin{bf} \columncolor{lightblue}} p{7cm} <{\end{bf}}|p{7cm}|}
  \hline
  Entité & Avion \\ 
  \hline
  Type d'Entité & Temporaire \\ 
  \hline
  Variables d'états & status  \\
  \hline
  Variables statistiques de scrutation & Aucune\\ 
  \hline
   & ID \\
    \multirow{-2}{7cm}{Paramètres techniques et données d'initialisation }& type \\ 
  \hline
   & EvtOncomingPlane\\
   & EvtApproach \\
   & EvtLanding \\
   & EvtRollIn\\
   & EvtUnload \\
   & EvtRefueling\\
   & EvtLoading \\
   & EvtRollOut \\
   \multirow{-9}{7cm}{Événements}& EvtTakeOff \\ 
  \hline
   & Process de durée de roulement à l'arrivée \\
   & Process de durée d'approche pour l'atterrissage \\
   \multirow{-3}{7cm}{Processus Stochastique} & Process de durée de roulement au départ\\ 
   \hline
\end{tabular}
\end{center}
\caption{Analyse de l'entité avion}
\label{avionAna}
\end{table}


Les variables d'initialisation portent sur :
\begin{itemize}[label=--]
\item le numéro d'identification ID
\item le type d'entité. Il a la même valeur pour tous les avions. Il permet d'identifier cette entité comme étant bien un avion
\end{itemize}
Les états de l'avion sont :
\begin{itemize}[label=--]
\item En attente de libération de la piste et du taxiway pour atterrir
\item En approche
\item En cours d'atterrissage
\item En roulement sur le taxiway 1
\item En attente sur le taxiway 1
\item En cours de déchargement 
\item En cours de réapprovisionnement
\item En cours de chargement
\item En attente de libération du taxiway 2
\item En roulement sur le taxiway 2
\item En attente sur le taxiway 2
\item En cours de décollage
\end{itemize}
Les événements temporels identifiés sont :
\begin{itemize}[label=--]
\item L'avion contacte la tour pour signifier son approche
\item L'avion réalise la phase d'approche de la piste
\item Atterrissage de l'avion
\item Roulement de l'avion en entrée sur le taxiway 1
\item Le déchargement de l'avion
\item Le ravitaillement de l'avion
\item Le chargement de l'avion
\item Le roulement de l'avion en sortie sur le taxiway 2
\item Le décollage de l'avion
\end{itemize}

La durée de certains événements est déterminée par un processus stochastique, conformément à ce qui a été demandé dans le cahier des charges. Le roulement en sortie et en entrée prends entre 2 et 6 minutes. La durée d'approche pour l'atterrissage prends entre 2 et 5 minutes par beau temps et le double dans le cas contraire.

\section{Les infrastructures}

\begin{table}[H]
\begin{center}
\begin{tabular}{|>{\begin{bf} \columncolor{lightblue}} p{7cm} <{\end{bf}}|p{7cm}|}
  \hline
  Entité & infrastructures \\ 
  \hline
  Type d'Entité & Permanente \\ 
  \hline
    & runway.status \\
    & taxiway1.status \\
    & taxiway2.status \\
    \multirow{-4}{7cm}{Variables d'états} & gates[k].status (k de 0 à nb\_gate) \\ 
  \hline
  Variables statistiques de scrutation  & Aucune \\ 
  \hline
  Paramètres techniques et données d'initialisation  & Aucun \\ 
  \hline
   & EvtRelease\_P\_TW1 \\
   & EvtReleaseGate \\
   & EvtRelease\_TW2 \\
   \multirow{-4}{7cm}{Événements}&  EvtRelease\_P  \\ 
  \hline
    Comportement & Aucun \\ 
  \hline
   Processus Stochastique & Aucun\\ 
   \hline
\end{tabular}
\end{center}
\caption{Analyse de l'entité infrastructure}
\label{infraAna}
\end{table}

Il n'y a pas de paramètres d'initialisation des infrastructures. Elles sont toujours composées d'une piste, de deux taxiway et d'une liste de portes dont le nombre est défini à la création de l'aéroport.

Les états des infrastructures sont accessibles via le champs status des entités filles qui la composent.

Les événements temporels identifiés sont:
\begin{itemize}[label=--]
\item La libération de la piste et du taxiway 1
\item La libération d'une porte
\item La libération du taxiway 2
\item La libération de la piste seule
\end{itemize}

\section{Le scénario}

\begin{table}[H]
\begin{center}
\begin{tabular}{|>{\begin{bf} \columncolor{lightblue}} p{7cm}<{\end{bf}}|p{7cm}|}
  \hline
  Entité &  agenda\\ 
  \hline
  Type d'Entité & Permanente \\ 
  \hline
  Variables d'états & Aucune \\ 
  \hline
  Variables statistiques de scrutation & offset \\ 
  \hline
    & startDate\\
    \multirow{-2}{7cm}{Paramètres techniques et données d'initialisation } & stopDate \\ 
  \hline
   Événements & EventsInitializer \\ 
  \hline
    & generatePlane \\
    \multirow{-2}{7cm}{Comportement } & generateMeteo\\ 
  \hline
   & Process de génération de l'arrivée des avions\\
   \multirow{-2}{7cm}{Processus Stochastique} & Process de génération de la météo \\ 
   \hline
\end{tabular}
\end{center}
\caption{Analyse de l'entité aéroport}
\label{scenaAna}
\end{table}

Les variables d'initialisation portent sur:
\begin{itemize}[label=--]
\item La date de début de simulation
\item La date de fin de simulation
\end{itemize}

Le seul événement temporel identifié est l'événement d'initialisation du scénario qui va générer les changements de météo ainsi que les arrivées d'avions. Ceci est effectué par un processus stochastique.
