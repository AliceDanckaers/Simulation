%Suggestions d’amélioration du logiciel de simulation, du modèle… 

\section{Amélioration de la simulation}
Une des pistes d'amélioration du logiciel de simulation serait de mettre en place un logger plus poussé. En effet le logger actuel crée un fichier Excel "brut", qu'il convient de retraiter pour en extraire les données statistiques, ce qui est particulièrement pénible, notamment lorsque l'on a eu aucune formation en Excel. Un logger plus poussé contiendrait déjà les formules permettant de calculer les résultats demandés sans manipulation supplémentaire, ou très peu.
Une autre amélioration serait d'améliorer la modularité du la simulation, d'autoriser plus de réglages différents, notamment au niveau du nombres des infrastructures (pistes, etc) et des fréquences d'arrivées des avions.

\section{Amélioration du modèle}
Quelques améliorations pourraient rendre le modèle plus robuste. Par exemple, il faudrait mettre en place une gestion du détournement des avions : dans le cas où un avion reste en attente au-delà d'un certain temps, il devrait être redirigé vers un autre aéroport et ne plus tourner dans l'espace aérien sur des réserves de carburant infinies.
Ces détournements pourraient même advenir de façon préventive, si l'aéroport est saturé, sans que l'avion n'ai a patienter en vol avant de partir vers un autre aéroport.



